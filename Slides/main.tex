%\documentclass[t,handout,xcolor=dvipsnames]{beamer}
\documentclass{beamer}
\usepackage{tikz}
\usepackage{import}
\usepackage{docmute}
% \usepackage{enumitem}
\usepackage{animate}
\usepackage{listings}
\usepackage[T1]{fontenc}
\usepackage[edges]{forest}
\usepackage[utf8]{inputenc}

\usetheme{Madrid}
\usecolortheme{default}

\usetikzlibrary{tikzmark,decorations.pathreplacing,shadows,arrows.meta}
\tikzstyle{curly} = [decorate,decoration={brace,amplitude=10pt}]
\colorlet{ProcessBlue}{blue!50!cyan}

\tikzset{
  selected/.style={draw=ProcessBlue, thick, rounded corners=2pt, inner color=ProcessBlue!25, outer color=ProcessBlue!35, drop shadow,},
}

\AtBeginSection[]{
  \begin{frame}
  \vfill
  \centering
  \begin{beamercolorbox}[sep=8pt,center,shadow=true,rounded=true]{title}
    \usebeamerfont{title}\insertsectionhead\par%
  \end{beamercolorbox}
  \vfill
  \end{frame}
}

\centering
\title{Analysis \& Modelling of Complex Data (1\textsuperscript{st} Half)}
\author{Aldo Faisal \& Chaiyawan Auenpanwiriyakul}
\institute{Universität Bayreuth}
\date{April 19, 2024}

\lstset{
  language=Python,
  basicstyle=\ttfamily\small,
  keywordstyle=\color{blue},
  stringstyle=\color{red},
  commentstyle=\color{green},
  showstringspaces=false,
  numbers=none,
  texcl=true,
  numberstyle=\tiny,
  breaklines=true,
  breakatwhitespace=true,
  frame=single,
  columns=fullflexible,
  keepspaces=true,
  captionpos=b,
  upquote=true
}


\begin{document}

\frame{\titlepage}

\section{Outline}
\begin{frame}[allowframebreaks]
\frametitle{Outline}
\tableofcontents
\end{frame}

\subimport{01 Course Introduction}{chapter.tex}
% \graphicspath{{01 Introduction/figures}}

% \todo[inline]{Move this to lab slides}

% \section{Our coding environment}
% \begin{frame}
% \frametitle{What is a Jupyter Notebook?}
% \begin{itemize}
% \item A Jupyter Notebook is an open-source web application
% \item Allows you to create and share documents containing live code, equations, visualizations, and narrative text
% \item Supports multiple programming languages, including Python, R, and Julia
% \item Ideal for interactive data analysis, visualization, and documentation
% \end{itemize}
% \begin{center}
% \includegraphics[width=0.33\linewidth]{jupyter-notebook-logo.png}
% \end{center}
% \begin{itemize}
% \item Accessible through a web browser
% \item Can be run locally or on a remote server
% \item Jupyter Notebooks have a .ipynb file extension
% \end{itemize}
% \end{frame}
% \begin{frame}

% \frametitle{What platform to use?}
% You have a choice:
% \begin{itemize}
% \item Google Colab:
% \begin{itemize}
% \item Cloud-based "Jupyter Notebook" environment
% \item Ideal for collaboration and easy access
% \item No software installation required; ready to go.
% \item You have to be online all the time.
% \end{itemize}
% \item Local Python Software Installation on your computer
% \begin{itemize}
% \item Install Python and required libraries on your computer
% \item Use a local Jupyter Notebook 
% \item Full control over your environment and packages
% \item Works offline
% \end{itemize}
% \item University installation of Python ...
% \end{itemize}
% \end{frame}

% \subsection{Introduction to Google Colab}
% \begin{frame}
% \frametitle{What is Google Colab?}
% \begin{itemize}
% \item Google Colab is a free, cloud-based Jupyter Notebook environment
% %\item Offers GPU and TPU support for faster computation
% \item Ideal for machine learning, data analysis, and collaboration
% \item No setup required runs entirely in your web browser
% \item Visit \url{https://colab.research.google.com/} to access Google Colab
% \item Sign in with your Google account
% \item Create a new notebook by clicking on File' $\rightarrow$ New notebook'
% \item Open an existing notebook from Google Drive or upload a local file
% \end{itemize}
% \end{frame}

% \begin{frame}
% \frametitle{Basic Colab Notebook Operations}
% \begin{itemize}
% \item Notebooks are composed of cells, which can be code cells or text (Markdown) cells
% \item Click on a cell to edit its content
% \item Press Shift + Enter to run a cell and move to the next one
% \item Add new cells using the + Code' and + Text' buttons
% \item Reorder cells by clicking and dragging the cell handle
% \end{itemize}
% \end{frame}

% %\begin{frame}
% %\frametitle{Using GPU and TPU Acceleration}
% %\begin{itemize}
% %\item Enable hardware acceleration for faster computation
% %\item Click on Runtime' $\rightarrow$ Change runtime type'
% %\item Select GPU or TPU from the Hardware accelerator' %dropdown menu \item Click Save to apply the changes
% %\end{itemize}
% %\begin{center}
% %\includegraphics[width=0.7\linewidth]{colab-runtime}
% %\end{center}
% %\end{frame}

% \begin{frame}
% \frametitle{Sharing and Collaborating with Google Colab}
% \begin{itemize}
% \item Share your notebook with others using the Share button in the top-right corner \item Control access levels (view, comment, or edit) for collaborators \item Work on the notebook simultaneously in real-time \item Leave comments for your collaborators by highlighting text and clicking the Comment button
% \end{itemize}
% \end{frame}

% \subsection{Installing Python and Using pip on a Windows Laptop}
% \begin{frame}
% \frametitle{Downloading Python}
% \begin{itemize}
% \item Visit the official Python website: \url{https://www.python.org/}
% \item Click on the Downloads tab and select Windows
% \item Download the latest Python version for Windows (Python 3.x)
% \end{itemize}
% \end{frame}

% \begin{frame}
% \frametitle{Installing Python}
% \begin{itemize}
% \item Run the downloaded installer
% \item Check the box Add Python to PATH' to automatically set up the environment variable \item Click on Customize installation' to choose your desired settings, or `Install Now' for default settings
% \item Wait for the installation to complete
% \end{itemize}
% \end{frame}

% \begin{frame}
% \frametitle{Verifying Python Installation}
% \begin{itemize}
% \item Open the Command Prompt or PowerShell
% \item Type `python --version' and press Enter
% \item If the installation was successful, you should see the Python version displayed
% \end{itemize}
% \end{frame}

% \begin{frame}
% \frametitle{Using pip to Manage Packages/Libraries}
% \begin{itemize}
% \item pip is the package installer for Python, included with Python 3.4 and later
% \item Open the Command Prompt or PowerShell
% \item Install a package using: pip install package-name
% \item Uninstall a package using: pip uninstall package-name
% \item List installed packages using: pip list
% \end{itemize}
% \end{frame}

\subimport{02 Introduction to Python}{chapter.tex}
% \graphicspath{{02 Introduction to Python/figures}}


% \subsection{Python Lists}
% \begin{frame}
% \frametitle{Introduction to Python Lists}
% \begin{itemize}
% \item Python lists are versatile, ordered data structures
% \item Lists can store elements of various data types, including numbers, strings, and other objects
% \item Lists are useful when you do not know how many elements you want to include or if the number may change dynamically, e.g. a to-do list.
% \end{itemize}
% \end{frame}


% \begin{frame}[fragile]
% \frametitle{Python Lists}
% \begin{lstlisting}[caption=Lists in Python][language=Python]

% # Creating a list

% numbers = [1, 2, 3, 4, 5]

% # Accessing list elements

% first_number = numbers[0]

% # Modifying list elements

% numbers[1] = 6
% \end{lstlisting}
% \end{frame}

% \section{Underconstruction}
% \begin{frame}[fragile]
% \frametitle{Underconstruction}
% \end{frame}

\subimport{03 Python and Statistics}{chapter.tex}

% \graphicspath{{03 Python and Statistics/figures}}

% \begin{frame}[fragile]{Scientific Frameworks and Libraries}
%     \textbf{matplotlib.pyplot}
%     \begin{tabular}{ p{5.7cm} p{5.7cm} }
%         \hline
%         \textbf{Functions} & \textbf{Description} \\
%         \hline
%         matplotlib.pyplot.\textbf{xlabel}(xlabel, fontdict=None, **kwargs) & Set the label for the x-axis. \\
%         matplotlib.pyplot.\textbf{ylabel}(ylabel, fontdict=None, **kwargs) & Set the label for the y-axis. \\
%         matplotlib.pyplot.\textbf{text}(x, y, s, fontdict=None, **kwargs) & Add text to the location. \\
%         matplotlib.pyplot.\textbf{grid}(**kwargs) & Show the axes grids. \\
%         matplotlib.pyplot.\textbf{subplot}(*args, **kwargs) & Add an Axes to the current figure or retrieve an existing Axes. \\
%         \hline
%     \end{tabular}
% \end{frame}

\subimport{04 Plotting with Python}{chapter.tex}

\subimport{05 Working with Big Data}{chapter.tex}


% \subsection{File format for Big Data}


% 4 weeks
% \section{Machine Learning}

% 2 weeks
% \subsection{Modelling Data with Modern Machine Learning}
% \begin{frame}{Under Construction}
    
% \end{frame}

\section{References}
\begin{frame}
\frametitle{References}
\begin{thebibliography}{9}
\bibitem{Python}
Official Python Website.
\url{https://www.python.org/}

\bibitem{NumPy}
NumPy Official Documentation.
\url{https://numpy.org/doc/stable/index.html}

\bibitem{Matplotlib}
Matplotlib Official Documentation.
\url{https://matplotlib.org/stable/contents.html}

\bibitem{hospital.csv}
Hospital Data in CSV file format.
\url{https://myfiles.uni-bayreuth.de/filr/public-link/file-download/ff80808284a840220184bdca018046e1/72460/5595612019581229783/hospital.csv}
\end{thebibliography}
\end{frame}
\end{document}