\usepackage{srcltx}
\usepackage{tikz}
\usepackage{tkz-graph}
\usetikzlibrary{decorations.markings}
\usetikzlibrary{arrows}
\usetikzlibrary{shapes}
\usetikzlibrary{automata}
\usetikzlibrary{plotmarks}
\usetikzlibrary{mindmap,trees,backgrounds}
\usetikzlibrary{positioning}
\usetikzlibrary{fit}
\tikzstyle{every picture}+=[remember picture]

\usepackage{listings}
\usepackage{pgfplots}
\pgfplotsset{compat=newest}
\setbeamertemplate{caption}[numbered]

\usepackage{multimedia}
%\usepackage{fancyhdr}
\usepackage{bm} % correct bold symbols, like \bm
\usepackage{color}
\usepackage{graphicx} % for pdf, bitmapped graphics files
\usepackage{natbib}

%\usepackage{ulem}
%\usefonttheme{professionalfonts}

\usepackage{multicol}

\usepackage{tikz-qtree,tikz-qtree-compat}

	\newcommand{\prob}[1]{\Pr\left[ #1 \right]}
	\newcommand{\Prob}[1]{P\left[ #1 \right]}
	\newcommand{\E}[1]{\mathds{E}\left[ #1 \right]}

% Common stuff for linear algebra
	\newcommand{\eye}{\mathbb I}
	\newcommand{\br}{{\bf r}}
	\newcommand{\bw}{{\bf w}}
	\newcommand{\bx}{{\bf x}}
	\newcommand{\by}{{\bf y}}
	\newcommand{\bz}{{\bf z}}
	\newcommand{\bp}{{\bf p}}
	\newcommand{\be}{{\bf e}}
	\newcommand{\bM}{{\bf M}}
	\newcommand{\bP}{{\bf P}}
	\newcommand{\bR}{{\bf R}}
	\newcommand{\bW}{{\bf W}}
	\newcommand{\bT}{{\bf T}}
	\newcommand{\bA}{{\bf A}}
	\newcommand{\bU}{{\bf U}}
	\newcommand{\pivec}{{\pi}}% for as bold face pi seems broken calculus
	\newcommand{\ddd}[2]{\frac{{\rm d}^{2} #1}{{\rm d} #2 ^{2}}}
	\newcommand{\dd}[2]{\frac{{\rm d} #1}{{\rm d} #2}}
	% Misc
	\newcommand{\ith}{^{\left( i \right)}}
	\newcommand{\te}{\!=\!} % thin equals
	\newcommand{\sS}{\mathcal{S}} % set of states
	\newcommand{\sA}{\mathcal{A}} % set of actions
%	\newcommand{\mdP[2]}{\mathcal{P}_{#1 #2}}
%	\newcommand{\mdR[2]}{\mathcal{R}_{#1 #2}}
	\newcommand{\mdP}[3]{\mathcal{P}_{#1 #2}^{#3}}
	\newcommand{\mdR}[3]{\mathcal{R}_{#1 #2}^{#3}}
	\newcommand{\mdE}[1]{\mathds{E}_\pi \left[ #1 \right]}


\renewcommand{\vec}[1]{{\boldsymbol{{#1}}}} % vector
\newcommand{\mat}[1]{{\boldsymbol{{#1}}}} % matrix
\newcommand{\tensor}[1]{\mathbb{#1}} % tensor
\newcommand{\R}[0]{\mathds{R}} % real numbers
\newcommand{\Z}[0]{\mathds{Z}} % integers
\newcommand{\tr}[0]{\text{tr}} % trace
\renewcommand{\d}[0]{\text{d}} % derivative
\newcommand{\GP}[0]{\mathcal{GP}} % Fourier transform
\newcommand{\e}[0]{\text{e}} % exp-function
\newcommand{\Set}[1]{{\cal #1}}
\newcommand{\erf}{\text{erf}}
\newcommand{\inv}{^{-1}}
\DeclareMathOperator*{\diag}{diag}
%\newcommand{\E}{\mathds{E}} % expectation
\newcommand{\var}{\mathds{V}}
\newcommand{\mean}[1]{\overline{#1}} % mean value
% \DeclareMathOperator{\d}{\mathrm{d}}
\newcommand{\data}{\mathcal D}
\newcommand{\gauss}[2]{\mathcal{N}\big(#1,\,#2\big)}
\newcommand{\gaussx}[3]{\mathcal{N}\big(#1\,|\,#2,\,#3\big)}
\newcommand{\gaussBig}[2]{\mathcal{N}\left(#1,\,#2\right)}
\newcommand{\gaussxBig}[3]{\mathcal{N}\left(#1\,|\,#2,\,#3\right)}
\DeclareMathOperator{\cov}{Cov}
\newcommand{\cost}{c}
\newcommand{\T}{^\top}
\newcommand{\polpar}{\theta}
\newcommand{\pol}{p_{\pi}}
\newcommand{\ppol}{p_{\pi^\prime}}

\makeatletter
\newcommand{\shorteq}{%
  \settowidth{\@tempdima}{-}% Width of hyphen
  \resizebox{\@tempdima}{\height}{=}%
}
\makeatother


\newcommand{\arrow}{
\begin{tikzpicture}
\draw [black!40!green, fill=black!40!green] (0,-0.12) -- (0,0.12) --
(0.15,0);
\draw [black!40!green, fill=black!40!green] (0.15,-0.12) -- (0.15,0.12) --
(0.3,0); 
\end{tikzpicture}
}

%\usefonttheme{professionalfonts}
\usetikzlibrary{decorations.fractals}
\usepackage{mathpazo}

%%% BEGIN BEAMER SPECIFIC CHANGES
\usetheme{JuanLesPins}  % 3D, Schatten
\usecolortheme{rose}
\useoutertheme{tree}
\newcommand<>\lightgray[1]{{\color#2[rgb]{0.8,0.8,0.8}#1}}

%\setbeamercolor{frametitle}{fg=white}
 \setbeamercolor{title in head/foot}{fg=lightgray}
 \setbeamercolor{section in head/foot}{fg=lightgray}
 \setbeamercolor{subsection in head/foot}{fg=lightgray}
\makeatletter
\setbeamertemplate{headline}
{%
    \begin{beamercolorbox}[wd=\paperwidth,colsep=1.5pt]{upper separation line head}
    \end{beamercolorbox}
    \begin{beamercolorbox}[wd=\paperwidth,ht=2.5ex,dp=1.125ex,%
      leftskip=.3cm,rightskip=.3cm plus1fil]{title in head/foot}
      \usebeamerfont{title in head/foot}\insertshorttitle   \hfill\insertframenumber/\inserttotalframenumber\hspace{0.5em}
    \end{beamercolorbox}
    \begin{beamercolorbox}[wd=\paperwidth,ht=2.5ex,dp=1.125ex,%
      leftskip=.3cm,rightskip=.3cm plus1fil]{section in head/foot}
      \usebeamerfont{section in head/foot}%
      \ifbeamer@tree@showhooks
        \setbox\beamer@tempbox=\hbox{\insertsectionhead}%
        \ifdim\wd\beamer@tempbox>1pt%
          \hskip2pt\raise1.9pt\hbox{\vrule width0.4pt height1.875ex\vrule width 5pt height0.4pt}%
          \hskip1pt%
        \fi%
      \else%  
        \hskip6pt%
      \fi%
      \insertsectionhead
      \usebeamerfont{subsection in head/foot}%
      \ifbeamer@tree@showhooks
        \setbox\beamer@tempbox=\hbox{\insertsubsectionhead}%
        \ifdim\wd\beamer@tempbox>1pt%
          \ \raise1.9pt\hbox{\vrule width 5pt height0.4pt}%
          \hskip1pt%
        \fi%
      \else%  
        \hskip12pt%
      \fi%
      \insertsubsectionhead
    \end{beamercolorbox}
    \begin{beamercolorbox}[wd=\paperwidth,colsep=1.5pt]{lower separation line head}
    \end{beamercolorbox}
}
\makeatother

\setbeamertemplate{blocks}[rounded][shadow=true]
%\setbeamertemplate{navigation symbols}{}
%\setbeamertemplate{headline}{}

%\setbeamerfont{myTOC}{series=\bfseries}%,size=\Large}

\newenvironment<>{myblock}[1]{%
  \begin{actionenv}#2%
      \def\insertblocktitle{#1}%
      \par%
      \mode<presentation>{%
       \setbeamercolor{block title}{fg=black,bg=blue!25!white}
       \setbeamercolor{block body}{fg=black,bg=gray!20}
       \setbeamercolor{itemize item}{fg=blue!40!white}
       \setbeamertemplate{itemize item}[triangle]
     }%
      \usebeamertemplate{block begin}
    {\par\usebeamertemplate{block end}}
 \end{actionenv}}

\newenvironment<>{myblock2}[1]{%
  \begin{actionenv}#2%
      \def\insertblocktitle{#1}%
      \par%
      \mode<presentation>{%
       \setbeamercolor{block title}{fg=white,bg=green!40!black}
       \setbeamercolor{block body}{fg=black,bg=gray!20}
       \setbeamercolor{itemize item}{fg=green!60!black}
       \setbeamertemplate{itemize item}[triangle]
     }%
      \usebeamertemplate{block begin}
    {\par\usebeamertemplate{block end}}
\end{actionenv}}
    
%%% END BEAMER SPECIFIC CHANGES

%\usepackage[T1]{fontenc}
% \usepackage[latin1]{inputenc}


\renewcommand{\emph}[1]{\textbf{{#1}}}
\newcommand{\argmax}[1]{\underset{#1}{\operatorname{argmax}}}

\newtheorem{remark}{Remark}
\newtheorem{assumption}{Assumption}
\renewcommand{\alert}[1]{{\red{{#1}}}}